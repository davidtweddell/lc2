\documentclass[border=40pt,tikz]{standalone}
\usepackage{graphicx}

\usepackage{subcaption}

% to use tikz
\usepackage{tikz}
\usetikzlibrary{mindmap, positioning, shapes.geometric, calc}
\usetikzlibrary{shapes.multipart, shapes.arrows, arrows.meta}
\usetikzlibrary{shadows, fit, decorations.markings}
\usetikzlibrary{backgrounds,calc}

% \usepackage{smartdiagram}

% to use custome fonts
\usepackage[no-math]{fontspec} 
% \setmainfont[BoldFont = {Roboto Bold}]{Roboto Bold}
\setmainfont{Aptos}[BoldFont = Aptos-Bold]
% \setmainfont{Roboto-Regular}[BoldFont = Roboto-Bold]
\setmonofont{IBMPlexMono-Regular}[BoldFont = IBMPlexMono-Bold]

% to manipulate colors
\usepackage{xcolor}
% \usepackage{pgfgantt}
\usepackage{mycolor}

% to read datafiles
\usepackage{datatool}

% page geometry
% \usepackage[left=1.50cm, right=1.50cm, top=1.50cm, bottom=2.00cm]{geometry}
% % headers and footers


%--------------------------------
%--------------------------------
\begin{document}
%--------------------------------
%--------------------------------

% %--------------------------------
% % define tikz formats
% %--------------------------------
% \tikzset{%
%         customdash/.style={dash pattern=on 2pt off 3pt on 4pt off 4pt},
%         tbcdash/.style={dash pattern=on 5pt off 5pt on 5pt off 5pt},
%         font={\fontsize{24pt}{24}\selectfont}
% }

% TODO: move frame labels to top right and top left corner

% \input{colourdefs.tex}
\definecolor{clustercolour1}{RGB}{251.999925, 79.00002, 47.999925000000005}
\definecolor{clustercolour2}{RGB}{228.999945, 174.000015, 56.00004}
\definecolor{clustercolour3}{RGB}{109.000005, 144.00003, 79.00002}
\definecolor{clustercolour4}{RGB}{138.99999, 138.99999, 138.99999}
\definecolor{clustercolour5}{RGB}{22.99998, 189.99999, 207.00007499999998}
\definecolor{clustercolour6}{RGB}{147.99996000000002, 103.00011, 188.99988}

\definecolor{featurecolour1}{RGB}{170.99993999999998, 114.000045, 0.0}
\definecolor{featurecolour2}{RGB}{205.999965, 133.000095, 255.0}
\definecolor{featurecolour3}{RGB}{0.0, 102.0, 119.99994}
\definecolor{featurecolour4}{RGB}{200.00007, 130.00002, 110.99997}
\definecolor{featurecolour5}{RGB}{82.99995, 68.99994, 138.99999}
\definecolor{featurecolour6}{RGB}{100.000035, 68.000085, 96.99996}
\definecolor{featurecolour7}{RGB}{235.000095, 207.99993, 154.999965}
\definecolor{featurecolour8}{RGB}{149.00007, 211.00000500000002, 255.0}
\definecolor{featurecolour9}{RGB}{204.0, 204.0, 204.0}


% angular position of labels, etc.
\def\startpos{90}

% radius of the circle
\def\circrad{5}

% radius for labels
\def\lblrad{\circrad + 1.0}
\def\lblindent{+10}
\def\lblsze{\large}
\def\lblsze{\fontsize{12pt}{24}\selectfont}
\def\ttlsze{\Large}

\def\boxsze{7}

% # arc features
\def\arcgap{2} % degrees to cut into the start and end positions
\def\arcwidth{0.5mm} % width of the arc line - the middle is set elsewhere

% size of nodes on joining lines
\def\noderad{.25}

% nodes
\DTLloaddb{clusternodes}{./nodes-cluster.csv}
\DTLloaddb{symptomnodes}{./nodes-symptom.csv}
\DTLloaddb{sitenodes}{./nodes-sites.csv}
% edges
\DTLloaddb{theedges}{./edges-cluster-x-symptoms.csv}
\DTLloaddb{siteclusteredges}{./edges-sites-x-clusters--2.csv}
\DTLloaddb{sitesymptedges}{./edges-sites-x-symptoms.csv}

% define some useful styles
\tikzset{
    dot/.style={draw, circle, fill=black, inner sep=0pt, minimum size=28pt},
    chord/.style={
            white, 
            line width=0.5mm, 
            double=gray, 
            double distance=0.5mm, 
    },
}


% layers
\pgfdeclarelayer{nodes}
\pgfdeclarelayer{edges}
\pgfsetlayers{main,edges,nodes}

% #================================
% CLUSTERS x SYMPTOMS
% #================================
\begin{tikzpicture}[double distance = 3mm, radius = \circrad]

% background rectangles
\draw[black, fill = black!10] (-\boxsze,-\boxsze) rectangle (0,\boxsze);
\draw[black, fill = red!10] (0,-\boxsze) rectangle (\boxsze,\boxsze);
       
% place text at the top of the box
\node[anchor=west, text width=5cm, align = left] at (-\boxsze,6.5) {\ttlsze \textbf{Symptom Clusters}};
\node[anchor=east, text width=5cm, align = right] at (\boxsze,6.5) {\ttlsze \textbf{Symptoms}};
    
% create the coordinates
\DTLforeach{clusternodes}{%
\MID=MID,\NAME=NODENAME}{%
    \coordinate (\NAME) at (\startpos + \MID:\circrad);
}

\DTLforeach{symptomnodes}{%
\MID=MID,\NAME=NODENAME}{%
    \coordinate (\NAME) at (\startpos + \MID:\circrad);
}


%  draw the edges
\begin{pgfonlayer}{edges}
    \DTLforeach{theedges}{%
\NSTART=NODESTART,\NEND=NODEEND,\WIDTH=WIDTH,\ES=EDGESTRING,\CLR=EDGECOLOUR}{%
    \draw[chord, double distance =\WIDTH, double=\CLR] \ES;}
\end{pgfonlayer}{edges}

% draw the nodes
\begin{pgfonlayer}{nodes}
    % \clip (0,0) circle (\circrad+2);

\DTLforeach{symptomnodes}{%
\START=START,\END=END,\MID=MID,\COLOUR=COLOUR,\LABEL=LABEL,\NAME=NODENAME,\WIDTH=NW}{%
  % generic
    % \draw (\startpos + \MID:\circrad) node[dot,text=white]{};
        \draw[black, line width=\arcwidth, double=\COLOUR] 
                (\startpos + \START + \arcgap:\circrad) 
                arc 
                (\startpos + \START + \arcgap:\startpos + \END - \arcgap:\circrad);
        \draw(\startpos + \MID:\lblrad) node[rotate=0, align = center, text width = 1.5 cm]   {\lblsze \LABEL};
}        
% draw the nodes
\DTLforeach{clusternodes}{%
\START=START,\END=END,\MID=MID,\COLOUR=COLOUR,\LABEL=LABEL,\NAME=NODENAME,\WIDTH=NW}{%
  % generic
        \draw[black, line width=\arcwidth, double=\COLOUR] 
                (\startpos + \START + \arcgap:\circrad) 
                arc 
                (\startpos + \START + \arcgap:\startpos + \END - \arcgap:\circrad);
        \draw(\startpos + \MID:\lblrad) node[rotate=0, align = left, text width = 2cm]   {\lblsze \LABEL};
        
        }        
\end{pgfonlayer}{nodes}

\end{tikzpicture}

%=-------------------------------
%=-------------------------------
% CLUSTERS x SITES
%=-------------------------------
%=-------------------------------
\begin{tikzpicture}[double distance = 3mm, radius = \circrad]

    % background rectangles
    \draw[black, fill = black!10] (-\boxsze, -\boxsze) rectangle (0,\boxsze);
    \draw[black, fill = blue!10] (0,-\boxsze) rectangle (\boxsze,\boxsze);
           
    % place text at the top of the box
    \node[anchor=west, text width=5cm, align = left] at (-\boxsze,6.5) {\ttlsze \textbf{Symptom Clusters}};
    \node[anchor=east, text width=5cm, align = right] at (\boxsze,6.5) {\ttlsze \textbf{Sites}};
        
    %--------------------------------
    % create the node coordinates
    %--------------------------------
    \DTLforeach{clusternodes}{%
    \MID=MID,\NAME=NODENAME}{%
    \coordinate (\NAME) at (\startpos + \MID:\circrad);
    }
    
    %--------------------------------
    % site coordinates - # better way to do this!
    % use the MIDALT coordinate because it's mirrored
    %--------------------------------
    \DTLforeach{sitenodes}{%
    \MID=MID,\NAME=NODENAME}{%
    \coordinate (\NAME) at (\startpos + -1*\MID:\circrad);
    }

    %--------------------------------
    % draw edges
    %--------------------------------
    \DTLforeach{siteclusteredges}{%
    \WIDTH=WIDTH,\ES=EDGESTRING,\CLR=EDGECOLOUR}{%
    \draw[chord, double distance = 2*\WIDTH/10, double=\CLR] \ES;}
    
    %--------------------------------
    % nodes - site
    %--------------------------------
    \DTLforeach{sitenodes}{%
    \SITE=SITE,\LBL=LABEL,\START=START,\END=END,\MID=MID}{%
            \draw[black, line width=\arcwidth, double=gray] 
                (\startpos + -1*\START - \arcgap:\circrad) 
                arc 
                (\startpos + -1*\START - \arcgap:\startpos + -1*\END + \arcgap:\circrad);
            \draw(\startpos + -1*\MID:\lblrad-0.0) node[rotate=0, align = left]   {\lblsze \LBL};
    }
    %--------------------------------
    % nodes - clusters
    %--------------------------------
    \DTLforeach{clusternodes}{%
    \START=START,\END=END,\MID=MID,\COLOUR=COLOUR,\LABEL=LABEL,\NAME=NODENAME,\WIDTH=NW}{%
            \draw[black, line width=\arcwidth, double=\COLOUR] 
                    (\startpos + \START + \arcgap:\circrad) 
                    arc 
                    (\startpos + \START + \arcgap:\startpos + \END - \arcgap:\circrad);
            \draw(\startpos + \MID:\lblrad) node[rotate=0, align = left, text width = 2cm]   {\lblsze \LABEL};
    }        
    
    \end{tikzpicture}
    
%================================
%================================
% SITES x SYMPTOMS
%================================
%================================
\begin{tikzpicture}[double distance = 3mm, radius = \circrad]

    % background rectangles
    \draw[black, fill = blue!10] (-\boxsze,-\boxsze) rectangle (0,\boxsze);
    \draw[black, fill = red!10] (0,-\boxsze) rectangle (\boxsze,\boxsze);
           
    % place text at the top of the box
    \node[anchor=west, text width=5cm, align = left] at (-\boxsze,6.5) {\ttlsze \textbf{Sites}};
    \node[anchor=east, text width=5cm, align = right] at (\boxsze,6.5) {\ttlsze \textbf{Symptoms}};
    
    %--------------------------------
    % create the coordinates
    %--------------------------------
    \DTLforeach{sitenodes}{%
    \MID=MID,\NAME=NODENAME}{%
    \coordinate (\NAME) at (\startpos + \MID:\circrad);
    }

    \DTLforeach{symptomnodes}{%
    \MID=MID,\NAME=NODENAME}{%
    \coordinate (\NAME) at (\startpos + \MID:\circrad);
    }
    
    %--------------------------------
    %  draw the edges
    %--------------------------------
    \DTLforeach{sitesymptedges}{%
    \WIDTH=WIDTH,\ES=EDGESTRING,\CLR=EDGECOLOUR}{%
    \draw[chord, double distance =\WIDTH, double=\CLR] \ES;}
    
    
    %--------------------------------
    % % draw the nodes
    %--------------------------------
    \DTLforeach{sitenodes}{%
    \SITE=SITE,\LBL=LABEL,\START=START,\END=END,\MID=MID}{%

    \def\rownum{\DTLcurrentindex}
      \draw[black, line width=\arcwidth, double=gray] 
      (\startpos + \START + \arcgap:\circrad) 
      arc 
      (\startpos + \START + \arcgap:\startpos + \END - \arcgap:\circrad);
    \draw(\startpos + \MID:\lblrad) node[rotate=0, align = left]   {\lblsze \LBL};

    }

    \DTLforeach{symptomnodes}{%
    \START=START,\END=END,\MID=MID,\COLOUR=COLOUR,\LABEL=LABEL,\NAME=NODENAME,\WIDTH=NW}{%
            \draw[black, line width=\arcwidth, double=\COLOUR] 
                    (\startpos + \START + \arcgap:\circrad) 
                    arc 
                    (\startpos + \START + \arcgap:\startpos + \END - \arcgap:\circrad);
            \draw(\startpos + \MID:\lblrad) node[rotate=0, align = center, text width = 1.5cm]   {\lblsze \LABEL};
    }        
    
    \end{tikzpicture}

%--------------------------------
%--------------------------------
\end{document}
%--------------------------------
%--------------------------------